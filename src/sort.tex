\section{The \emph{Sort} Operator} \label{sort}

Sorting is among the most commonly used operations in database systems, forming
the core of operators such as \emph{merge join}, \emph{order-by} and some
flavors of \emph{group-by}.  The process of sorting is quite write-intensive
since the commonly used in-memory sorting algorithms, such as
\textit{quicksort}, involve considerable data movement. In the single pivot
quicksort algorithm with $n$ elements, the average number of swaps is of the
order of $0.3nln(n)$~\cite{swaps}. There are other algorithms such as
\emph{selection sort} which involve much less data movement, but they incur
\emph{quadratic} time complexity in the number of elements to be sorted, and are
therefore unsuitable for large datasets.

The main advantage associated with the quicksort algorithm is that it has good
average-case time complexity and that it sorts the input data in-place. If the
initial array is much larger than the DRAM size, it would entail evictions from
the DRAM during the swapping process of partitioning. These evictions might lead
to PCM writes if the evicted DRAM lines are \textit{dirty}, which is likely
since elements are being swapped. If the resulting partition sizes continue to
be larger than DRAM, partitioning them in turn will again cause DRAM evictions
and consequent writes. Clearly, this trend of writes will continue in the
recursion tree until the partition sizes become small enough to fit within DRAM.
Thereafter, there would be no further evictions during swapping and the
remaining sorting process would finish inside the DRAM itself.

From the above discussion, it is clear that it would be desirable for the
sorting algorithm to converge fast to partition sizes below DRAM size with
fewer number of swaps. For uniformly-distributed data, these requirements are
satisfied by \emph{flashsort}~\cite{flashsort}. On the other hand, for data with
skewed distribution, we propose a variant of flashsort called \emph{multi-pivot
flashsort}. This algorithm adopts the pivot selection feature of the quicksort 
algorithm into flashsort in order to tackle the skewness in data.

Both these algorithms are discussed in detail in the following sections. 

\subsection{Data with uniform distribution}\label{sort_uniform} 

The flashsort algorithm can potentially form DRAM-sized partitions in a
\emph{single} partitioning step with at most $N_R$ swaps. The sorting is done
in-place with a time complexity of $O(N_Rlog_2N_R)$ with constant extra space.
The flashsort algorithm proceeds in three phases: \emph{Classification},
\emph{Permutation} and \emph{Short-range Ordering}. A brief description of each
of these phases is as follows:

\subsubsection{Classification phase} The classification phase divides the input
data into equi-range partitions comprising of contiguous and disjoint ranges.
That is, if p partitions are required (where p is an input parameter), the
difference between the minimum and the maximum input values is divided by p.
Subsequently, each tuple is mapped to a partition depending on in which range
the value of the sorting attribute of the tuple lies.  Specifically, a tuple
with attribute value $v$ is assigned to $Partition(v)$, computed as
$$Partition(v) = 1 + \lfloor \frac{(p-1)(v- v_{min})}{v_{max}-v_{min}} \rfloor$$
where $v_{min}$ and $v_{max}$ are the smallest and largest attribute values in
the array, respectively. The number of tuples in each such partition is counted
to derive the boundary information.  We choose the number of partitions $p$ to
be $\lceil c \times \frac{N_R L_R}{D} \rceil$, where $c \geq 1$ is a multiplier
to cater to the space requirements of additional data structures constructed
during sorting. In our experience, setting $c = 2$ works well in practice.

\subsubsection{Permutation phase} The Permutation phase moves the elements to
their respective partitions by leveraging the information obtained in the
Classification phase. The elements are swapped in a cyclic manner to place each
element inside its partition boundary with a single write step per element. 

\subsubsection{Short-range Ordering phase} The resulting partitions, each having
size less than $D$, are finally sorted in the Short-range Ordering phase using
quicksort. Note that, by virtue of their size, these partitions are not expected
to incur any evictions during the process of sorting.  \\

\textbf{PCM write analysis}: Though the partition boundary counters are
continuously updated during the Classification phase, they are expected to incur
very few PCM writes. This is because the updates are all in quick succession,
making it unlikely for the counters to be evicted from DRAM during the update
process. Next, while in the Permutation phase, there are no more than $N_R L_R$
writes since each tuple is written at most once while placing it inside its
partition boundaries. Since each partition is within the DRAM size, its
Short-range Ordering phase will finish in the DRAM itself, and then there will
be another $N_R L_R$ writes upon eventual eviction of sorted partitions to PCM. 

Thus, the number of word-writes incurred by this algorithm is estimated by
\begin{equation} \label{eq:sort} \FlashsortWrites = \dfrac{2N_RL_R}{4} = \dfrac{N_R
L_R}{2} \end{equation}

\subsection{Data with non-uniform distribution} \label{sort_non_uniform} 

In the case when the data is non-uniformly distributed, the equi-range
partitioning used by flashsort fails to produce equi-sized partitions. This is
because the number of tuples in each range is now dependent on the skew of the
data.  We therefore propose an alternative algorithm, called \emph{multi-pivot flashsort}, 
which uses multiple pivots
instead to partition the input tuples. These pivots are randomly-chosen from the input
itself, in the same manner as conventional quicksort selects a single pivot to
create two partitions. The chosen pivots are subsequently leveraged to partition the 
input during sorting.  

The modified phases of this alternative implementation of the flashsort algorithm, along 
with their pseudo-codes, are described next.

\subsubsection{Classification phase} In the Classification phase, we divide the
input relation into $p$ partitions, where $p = \lceil \frac{N_R L_R}{D}
\rceil$, using $p-1$ random tuples as pivots. Since the pivots are picked at
random, the hope is that each partition is approximately of size $D$.  These
pivots are then copied to a separate location and sorted. Subsequently, we scan
through the array of tuples in the relation, counting the number of elements
between each consecutive pair of pivots. This is accomplished by carrying out,
for each tuple in the array, a binary search within the sorted list of pivots.

In spite of the random choice of pivot values, it is quite possible that some
partitions may turn out to be larger than the DRAM. We account for this
possibility by conservatively creating a larger number of initial partitions.
Specifically, the number of partitions is $p = \lceil c \times \frac{N_R
L_R}{D} \rceil$, where $c \geq 1$ is a design parameter similar to the one used
in the flashsort algorithm. Subsequently, we consider each pair of adjoining
partitions and coalesce them if their total size is within the DRAM size, after
leaving some space for bookkeeping information. 

While the above heuristic approach is quite effective, it still does not
guarantee that all the resultant partitions will be less than DRAM size. The
(hopefully few) cases of larger-sized partitions are subsequently handled
during the Short-range Ordering phase.

The pseudo-code for the Classification phase is outlined in
Algorithm~\ref{alg:read_phase}.

\begin{algorithm}
\small
\caption{Classification Phase}
\label{alg:read_phase}
\textbf{array[]} is the array of input tuples\\
\textbf{c} is a design parameter $\geq 1$\\
\begin{algorithmic}[1]
\State p = $\lceil c\times \frac{N_R L_R}{D} \rceil$
\State randIndex[] = generate $p-1$ random indexes
\State pivot[] = array[randIndex];
\State sort(pivot[])
\State size[] = {0...0}   
\Comment{size of sub-arrays}
\State partitionStart[] = {0...0}
\Comment{starting offset of each partition}
\For {i=$1$ to $N_R$}
\State partition = getPartition(array[i]) 
\State size[partition]++ 
\EndFor
\Comment {Time complexity=$N_R\times log_2p$ }
\State cumulative = 0
\For {i=$1$ to $p$}
\State cumulative = cumulative + size[i]
\State partitionStart[i+1] = cumulative
\EndFor
\Comment {Time complexity=$p$ }
\State return partitionStart[]
\end{algorithmic}

\end{algorithm}


\subsubsection{Permutation phase} The Permutation phase uses the information
gathered in the Classification phase to group tuples of the same partition
together. A slight difference from flashsort here is that the attribute value
now needs to be compared against the sorted list of pivots to determine the
partition of the tuple.  The pseudo-code for the Permutation phase is shown in
Algorithm~\ref{alg:swap_phase}.  The maximum number of writes is bounded by
$N_R L_R$, corresponding to the worst case where \emph{every} tuple has to be
moved to its correct partition.

\begin{algorithm}[h!]
\small
\caption{Permutation Phase}
\label{alg:swap_phase}
\textbf{partitionStart[]} is obtained from Classification Phase\\
\textbf{nextUnresolvedIndex[]} indicates the next position to be examined for each partition\\
\begin{algorithmic}[1]
\State nextUnresolvedIndex[] = partitionStart[]
\For {i=1 to $N_R$}
	\State curPartitionCorrect = getPartition(array[i])
     \If {i between partitionStart[curPartitionCorrect] and  partitionStart[curPartitionCorrect+1]}                
     \State nextUnresolvedIndex[curPartitionCorrect] = i+1
	\State continue
     \Else
     		\State firstCandidateLoc = i
			\State presentCandidate = array[i]             
            \State flag = 1
            \While {flag} 
                \State targetPartitionStart = nextUnresolvedIndex[curPartitionCorrect]
                \State targetPartitionEnd = partitionStart[curPartitionCorrect + 1]
                \For {k=targetPartitionStart to targetPartitionEnd} 
                \State nextPartitionCorrect = getPartition(array[i])
                    \If {k between partitionStart[nextPartitionCorrect]                       and 
                    \State partitionStart[nextPartitionCorrect + 1]}
                        \State continue
                        
                    \ElsIf {k == firstCandidateLoc} 
                        \State flag = 0 \Comment Indicates it is a cycle
                    \EndIf
                    \State swap(presentCandidate, array[k])
                    \State nextUnresolvedIndex[curPartitionCorrect] = k+1
                    \State curPartitionCorrect = nextPartitionCorrect
                    
                    \State break
             
				\EndFor
			\EndWhile
	\EndIf
\EndFor
\Comment {Time complexity=$N_R\times log_2p$ }
\end{algorithmic}

\end{algorithm}

\subsubsection{Short-range Ordering phase} Finally, each of the partitions are
sorted separately using conventional quicksort to get the final PCM sorted
array. For partitions that turn out to be within the DRAM size, the Short-range
Ordering phase is completed using conventional quicksort. On the other hand, if some
larger-sized partitions still remain, we recursively apply the multi-pivot
flashsort algorithm to sort them until all the resulting partitions can fit inside
DRAM and can be internally sorted.

\begin{algorithm}
\small
\caption{Short-range Ordering Phase}
\label{alg:sort_phase}
\begin{algorithmic}[1]
\For {i=1 to p}
\If {size[p] $<$ D}
\State quicksort (partition p)
\Else 
\State multi-pivot flashsort (partition p)
\EndIf 
\EndFor
\end{algorithmic}

\end{algorithm}

Figure~\ref{fig:mpsort} visually displays the steps involved in the multi-pivot
flashsort of an array of nine values. First, in the Classification phase, $30$
and $10$ are randomly chosen as the pivots. These pivots divide the input
elements into 3 different ranges: ($< 10$), ($\geq 10$, $< 30$), ($\geq 30$).
The count of elements in each of these ranges is then determined by making a
pass over the entire array -- in the example shown, three elements are present
in each partition.  Then, in the Permutation phase, the elements are moved to
within the boundaries of their respective partitions. Finally, in the
Short-range Ordering phase, each partition is separately sorted within the
DRAM.

\begin{figure}[h]
	\centering
	\subfloat[Classification Phase]{	
  	%\includegraphics[width=8cm]{sort_step_1.png}
  	
  	\begin{tabular}{|c|c|c|y|c|y|c|c|c|}
        \hline
        12&3&33&30&11&10&7&32&8\\\hline
    \end{tabular}
    }
	\hspace{0mm}
	\subfloat[Permutation Phase]{
  	\begin{tabular}{|x|x|x|y|y|y|z|z|z|}
        \hline
        7&3&8&12&11&10&30&32&33\\\hline
    \end{tabular}
    }
    \hspace{0mm}
	\subfloat[Short-range Ordering Phase]{
  	\begin{tabular}{|x|x|x|y|y|y|z|z|z|}
        \hline
        3&7&8&10&11&12&30&32&33\\\hline
    \end{tabular}
    }
	\caption{Multi-Pivot Flashsort}
	\label{fig:mpsort}
	
\end{figure}

%\newcolumntype{b}{>{\columncolor{white}}c}
\textbf{PCM write analysis}: The analysis follows that of the flashsort
algorithm. A negligible number of writes would be incurred during the copying and
sorting of the pivots. As mentioned, the writes during Permutation phase would 
be below $N_R L_R$. The creation of additional partitions by choosing extra
pivots, and their subsequent coalescing, increases the likelihood that each
partition is below DRAM size--akin to that in flashsort. Therefore
the total word-writes is again estimated to be \begin{equation}
\label{eq:mpivot} \MPFlashsortWrites = \frac{N_RL_R}{2} \end{equation}
